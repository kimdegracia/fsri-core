\documentclass[12pt,oneside]{book}
\input{commoncommands}

%rename chapter headings
\renewcommand{\chaptername}{}
\renewcommand{\bibname}{References}

\pagestyle{fancy}
\lhead{}
\rhead{}
\chead{}
\renewcommand{\headrulewidth}{0pt}

% COMMENT TO REMOVE WATERMARK
\usepackage{draftwatermark}
\SetWatermarkText{DRAFT}
\SetWatermarkScale{1}

\begin{document}
\pagenumbering{gobble}
\bibliographystyle{unsrt}
	
\begin{minipage}[t][9in][s]{6.25in}

\headerB{
Cleverly Named Title \\
To Describe Our Awesome \\ 
Work\\
}

\normalsize
\headerC{
	\flushleft{
	Author 1 \\
	Author 2 \\
	\vspace{0.2in}
	UL Firefighter Safety Research Institute \\
	Columbia, MD 20145 \\
	\vspace*{2\baselineskip}
	} 

	\vfill

	\flushright{
	\includegraphics[width=2.0in]{FSRI_GraphicShield} \\ [.3in]
	}
}

\footnotesize
\flushright{\textcopyright Underwriters Laboratories Inc. All rights reserved. UL and the UL logo are trademarks of UL LLC}
\end{minipage}

\newpage
\hspace{5in}
\newpage

\begin{minipage}[t][9in][s]{6.25in}
\pagenumbering{gobble}

\headerB{
Cleverly Named Title \\
To Describe Our Awesome \\ 
Work\\
}

\headerC{
	\flushleft{
	Author 1 \\
	Author 2 \\
	\vspace{0.2in}
	{UL Firefighter Safety Research Institute \\
	Columbia, MD 21045\\}
	}

	\flushleft{\today \\}
}

\vfill

\flushright{\includegraphics[width=2in]{FSRI_GraphicShield}}

\titlesigs

\footnotesize
\flushright{\textcopyright Underwriters Laboratories Inc. All rights reserved. UL and the UL logo are trademarks of UL LLC}
\end{minipage}

\newpage

\begin{minipage}[t][9in][s]{6.25in}

\flushleft{In no event shall UL be responsible to anyone for whatever use or non-use is made of the information contained in this Report and in no event shall UL, its employees, or its agents incur any obligation or liability for damages including, but not limited to, consequential damage arising out of or in connection  with the use or inability to use the information contained in this Report. Information conveyed by this Report applies only to the specimens actually involved in these tests. UL has not established a factory Follow-Up Service Program to determine the conformance of subsequently produced material, nor has any provision been made to apply any registered mark of UL to such material. The issuance of this Report in no way implies Listing, Classification or Recognition by UL and does not authorize the use of UL Listing, Classification or Recognition Marks or other reference to UL on or in connection with the product or system.
}

\vspace{3in}
\vfill
\hspace{1in}

\end{minipage}

\newpage

\frontmatter

\pagestyle{plain}
\pagenumbering{roman}

\cleardoublepage
\phantomsection
\tableofcontents

\cleardoublepage
\phantomsection
\addcontentsline{toc}{chapter}{List of Figures}
\listoffigures

\cleardoublepage
\phantomsection
\addcontentsline{toc}{chapter}{List of Tables}
\listoftables

\chapter{List of Abbreviations}

\begin{tabbing}
\hspace{1.5in} \= \\
UL 	    \> Underwriters Laboratories \\
UL FSRI \> UL Firefighter Safety Research Institute \\
\end{tabbing}

\mainmatter

\chapter*{Acknowledgments}
\pagenumbering{gobble}

\begin{table}[!ht]
	\centering
	\caption*{Generic Table}
	\begin{tabular}{ll}
		\toprule[1.5pt]
		Name & Awesomeness \\ 
		\midrule
		Neil Armstrong		 & First person to walk on the moon \\
		Edith Clarke		 & First female ECE professor \\
		John Goodenough		 & Rechargeable Lithium-Ion battery \\
		Nikola Telsa		 & Modern alternating current \\  
		\bottomrule[1.25pt]
	\end{tabular}
\end{table}

\newpage

\chapter*{Abstract}

\newpage
\chapter{Introduction}
\label{chap:intro}
\pagenumbering{arabic}
\setcounter{page}{1}

\section{Motivation}

\section{Objectives}


\chapter{Experimental Configuration}
\label{chap:exp_config}

\section{Instrumentation}
\label{sec:instrument}

\subsection{Measurement Locations}
\label{subsec:measure_locs}

\subsection{Measurement Uncertainty}
\label{subsec:uncertainty}

*** THIS LAYOUT CONTAINS UNCERTAINTY INFORMATION FOR A VARIETY OF DIFFERENTLY MEASUREMENT TYPES, SO BE SURE TO REMOVE ANY MEASUREMENT TYPES THAT AREN'T MENTIONED IN REPORT. ***

There are different components of uncertainty in the length, mass, temperature, heat flux, gas concentration, differential pressure, gas velocity, heat release rate, structure leakage, and water flow rate values reported throughout this document. Uncertainties are grouped into two categories according to the method used to estimate them. Type A uncertainties are those evaluated by statistical methods, and Type B are those evaluated by other means~\cite{Taylor&Kuyatt:1994}. Type B analysis of systematic uncertainties involves estimating the upper (+~a) and lower (-~a) limits for the quantity in question such that the probability that the value would be in the interval ($\pm$~a) is essentially 100~\%. After estimating uncertainties by either Type A or B analysis, the uncertainties are combined in quadrature to yield the combined standard uncertainty. Then, the combined standard uncertainty is multiplied by a coverage factor of two, which results in an expanded uncertainty with a 95~\% confidence interval (2$\sigma$). For some of these components, such as the zero and calibration elements, uncertainties are derived from referenced instrument specifications. For other components, referenced research results and past experience with the instruments provided input in the uncertainty determination.

\subsubsection*{Length Dimensions}
Length measurements, such as the room dimensions and instrumentation locations, were made with either a hand held laser measurement device having an accuracy of $\pm$~6.0~mm (0.25~in.) over a range of 0.61~m (2.0~ft) to 15.3~m (50.0~ft)~\cite{StanleyTools} or $\pm$~0.5~mm (0.02~in.) resolution steel measuring tapes manufactured in compliance with National Institute of Standards and Technology (NIST) Manual 44~\cite{Butcher:2012}, which specifies a tolerance of $\pm$~1.6~mm (0.06~in.) for 9.1~m (30~ft) tapes and $\pm$~6.4~mm (0.25~in.) for 30.5~m (100~ft) tapes. These uncertainties are all well within the precision of the reported dimensions, which are typically rounded to the nearest 0.1~m. Some issues, such as levelness of the device and ``soft'' edges on the upholstered furniture, result in an estimated expanded uncertainty of $\pm$~1.0~\% for reported length measurements.

\subsubsection*{Fuel Weights}
The load cell used to weigh the fuels prior to the experiments had a range of 0~kg (0~lb) to 200~kg (440~lb) with a resolution of a 0.05~kg (0.11~lb) and a calibration uncertainty within 1~\%~\cite{Ohaus:2000}. The total expanded uncertainty for the fuel weights presented in this report that were measured by the load cell is estimated to be less than $\pm$~5~\%.

\subsubsection*{Thermocouples}
For the bare-bead thermocouples used during the experiments, the standard uncertainty in the temperature of the thermocouple wire itself was reported by the manufacturer as being $\pm$~2.2~$^{\circ}$C at 277~$^{\circ}$C and increasing to $\pm$~9.5~$^{\circ}$C at 871~$^{\circ}$C~\cite{Omega:2004}. The variation of the temperature in the environment surrounding the thermocouple is known to be much greater than that resulting from the wire uncertainty. Expanded uncertainties as high as 20~\% for upper layer temperatures measured by a 1 mm bare-bead type K thermocouple have been reported by researchers at NIST~\cite{Blevins:1999,Pitts:2003}. Small diameter (approximately 0.25~mm?) thermocouples were used during these experiments to limit the impact of radiative heating and cooling. The total expanded uncertainty associated with the temperature measurements from these experiments is estimated to be $\pm$~15~\%.

\subsubsection*{Heat Flux Gauges}
Total heat flux measurements were made using water-cooled Schmidt-Boelter heat flux gauges. The manufacturer reports a $\pm$~3~\% calibration expanded uncertainty for these devices~\cite{Medtherm:2003}. Results from an international study on total heat flux gauge calibration and response demonstrated that the total expanded uncertainty of a Schmidt-Boelter gauge is typically $\pm$8~\%~\cite{Pitts:2006}.

\subsubsection*{Oxygen Concentration}
The gas sampling instruments used throughout the series of tests discussed in this report have demonstrated a relative expanded uncertainty of $\pm$~1~\% when compared to span gas volume fractions~\cite{Bundy:2007}. According to a study by Lock~et~al.~\cite{Lock:1}, the non-uniformities and movement of exhaust gases in addition to the limited amount of sampling points considered in each experiment result in an estimated expanded uncertainty of $\pm$~12~\%.

% The gas measurement instruments and sampling system used in this series of experiments have demonstrated an expanded (k~=~2) relative uncertainty of $\pm$~1~\% when compared with span gas volume fractions~\cite{Bundy:2007}. Given the non-uniformities and movement of the fire gas environment and the limited set of sampling points in these experiments, an estimated uncertainty of $\pm$~12~\% is applied to the results~\cite{Lock:1}.

\subsubsection*{Pressure Transducers}
The uncertainty components of differential pressure readings were derived from pressure transducer instrument specifications and previous experience with pressure transducers. The transducers were factory calibrated, and the zero and span of each were checked in the laboratory prior to the experiments, yielding an accuracy of $\pm$~1~\%~\cite{Setra:2002}. The total expanded uncertainty associated with pressure measurements obtained from the transducers is estimated as $\pm$~10~\%.

% Differential pressure reading uncertainty components were derived from pressure transducer instrument specifications and previous experience with pressure transducers. Each transducer was factory calibrated by the manufacturer to verify that the zero and span of the transducer resulted in an accuracy of $\pm$~1~\%~\cite{Setra:2002}. The total expanded uncertainty associated with the pressure data from these experiments is estimated to be $\pm$~10~\%.

\subsubsection*{Bi-Directional Probes}
Bi-directional probes paired with single type K, inconel-sheathed thermocouples were used to measure gas velocity. The bi-directional probes used pressure transducers similar to those used to measure differential pressure as discussed in the previous subsection. A gas velocity measurement study that focused on flow through doorways during pre-flashover compartment fires yielded total expanded uncertainties ranging from $\pm$~14~\% to $\pm$~22~\% for measurements from BDPs similar to those used throughout the experiments described in this report~\cite{Bryant:FSJ2009}. The total expanded uncertainty for gas velocity measured during these experiments is estimated to be $\pm$~18~\%. 

\subsubsection*{Heat Release Rate}

\subsubsection*{Leakage} 
To characterize ventilation within the structure, an air leakage measurement system was used to measure the amount of leakage associated with the training prop before each test~\cite{retrotec:leakage}. ASTM~E~779, `Standard Test Method for Determining Air Leakage Rate by Fan Pressurization', was followed to determine the air changes per area and the equivalent leakage area~\cite{astm_e779}. The leakage values were in units of air changes per hour at 50~Pa (ACPH50). 

\subsubsection*{Water Flow Rate}
Water flow rate was measured with a 3.81~cm (1.5~in.) diameter electromagnetic flow meter. The meter consisted of stainless steel pipe lined with a non-conductive material. Energized coils on the outside of the non-conductive material impose a magnetic field across the pipe. When the conductive fluid, water, flowed across the magnetic field a voltage, proportional to flow velocity, was created. The manufacturer reports a $\pm$~0.25~\% calibration uncertainty for the accuracy of the measurement~\cite{Badger:2015}. 

\chapter{Experimental Procedure}
\label{chap:exp_procedure}
% Example of Table Definition
% \begin{table}[!ht]
% 	\centering
% 	\caption{Example of Style in which Table Caption Should be Written}
% 	\label{tab:<ref_name>}
% 	\begin{tabular}{cl}
% 		\toprule[1.5pt]
% 		<header_1> & <header_2> \\
% 		\midrule
% 		<row_1_content>     & <row_1_content>     \\
% 		<row_2_content>     & <row_2_content>     \\
% 		\bottomrule[1.25pt]
% 	\end{tabular}
% \end{table}

% Example of Figure Definition
% \begin{figure}[!ht]
% 	\centering
% 	\includegraphics[width=<set_width>]{<file_location>}
% 	\caption[Title of Figure for List of Figures]{Write the figure caption in sentence form. The caption should contain enough detail for the reader to thoroughly understand the figure. So, if the figure is a data plot, be sure to describe the meaning of the various line colors, styles, symbols, etc. used.}
% 	\label{fig:<ref_name>}
% \end{figure}


\chapter{Results \& Discussion}


\chapter{Summary}


\bibliography{UL_general,UL_FSRI}

\clearpage

\appendix
\captionsetup{list=no}

\end{document}
